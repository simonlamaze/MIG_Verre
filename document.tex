\documentclass[12pt, a4paper, french, BCOR = 0pt, DIV = 10]{scrartcl}
\usepackage{graphicx} % Required for inserting images
\usepackage{babel}
\usepackage[utf8]{inputenc}
\usepackage[T1]{fontenc}
\usepackage{graphicx}
\graphicspath{ {./images/} }

\usepackage{bigints}

\usepackage{pxfonts}

\title{MIG Verre - Modélisation des fours}
\author{\small{Simon Lamaze - Corto Beck - Girardet Grégoire - Fraenkel Paul}}

\begin{document}
    
    \maketitle
    
    \section{Introduction}
    \raggedright
    Ce mini-projet traite de la modélisation et de l'optimisation énergétique des fours à verre.  La compréhension des phénomènes physiques qui prennt place dans un four est essentielle à l'optimisation des procédés. \\ [0.5 cm]
    Voici un four à fusion:
    

    
    
    
    
    \section{ Déroulement de la simulation}
    \raggedright
    Tâche 1 : positionnement du problème : \\
    Conditions limites, conditions initiales, mise en équations \\[0.3 cm]
    Tâche 2: prise en main des outils de calcul\\
    Maillage, cluster, post-traitement, CIMLIB, vitesse des calculs \\ [0.3 cm]
    
    Tâche 3 : Calcul sur un cas à voûte chaude\\ 
    Obtenir un cas convergé, bilan énergétique \\ [0.3 cm]
    
    Tâche 4 : Calcul sur un cas à voûte plus froide \\ 
    Same\\ [0.3 cm]
    
    Tâche 5: Bilan du gain énergétique et des émissions carbone \\ [0.5cm]
    
    
    On  utilise le logiciel Paraview pour visualiser les résultats.
    Visualiser la propagation d'un champ pour définir le temps de résidence.
    Faire plusieurs calculs avec des tailles différentes.
    
    \section{Positionnement du problème}
    - Les équations que nous nous appliquerons à résoudre seront les suivantes : \\
    \subsection{ Equations de Navier-Stokes et thermique}
    - Les équations que nous nous appliquerons à résoudre seront les suivantes : \\
    \centering
    $$
    \left\{
    \begin{array} {ll} 
        \vec {\nabla}. \vec{V} = 0  ~~~~~~~~~  conservation~de~la~masse\\
        
        \rho_{0} \frac{D\vec{V}}{Dt} = -\rho_{0} \beta (T-T_{0}) \vec{g} + \vec {\nabla} . [ 2 \eta (T) \vec{D}] - \vec {\nabla} P  ~~~~~~~~~ conservation~du~moment \\   
        
        \vec{D} = \frac{1}{2} [\vec{\nabla} \vec{V} + (\vec{\nabla} \vec{V})^t ] \\
        \rho [ C_{p}+ \Delta H_{r} \frac{d\alpha}{dT}] \frac{DT}{Dt} = \vec {\nabla} .  (\lambda_{eq}(T) \vec{\nabla} T ) + \sigma_{e}(T) (\vec \nabla\phi)^2  ~~~~~~~~~ \acute equation~de ~la~chaleur \\
        
        \vec{\nabla} . (\sigma_{e} \vec{\nabla} \phi) = 0  ~~~~~~~~~ conservation~du~flux~ \acute electrique
        
        
    \end{array}
    \right. 
    $$
    Comme les transfert de chaleur sous formes radiatives sont très importants, on ne peut pas les négliger dans l'équation habituelle de diffusion de la chaleur: $ \rho C_{p} \frac{\partial T}{\partial t} = - \vec{\nabla} . (\lambda\vec{\nabla}T) $. Cependant, comme le milieu est semi-transparent, on peut modéliser l'ensemble des transfert en remplaçant $\lambda$ par $\lambda_{eq} = \lambda + \frac{16n^{2} \sigma T^{3}}{3\beta_{R}} $.
    \\ [0.5 cm]
    
    
    \subsection{Puissance électrique}
    \raggedright
    On se place dans le cas uniphasé, en réalité, dans l'industrie le régime est triphasé, plus de calculs en annexes. \\ [0.2 cm ]
    \centering
    \[ P_{eq}=\int_\Omega \sigma_{e}(T)(\vec \nabla\phi)^2~dV  = \int_\Omega \vec \nabla[\sigma_{e}(T)\vec \nabla\phi]~dV \]\\
    \raggedright
    par intégration par partie et loi de conservation du potentiel ($ \vec{\nabla}^2 \phi = 0 $). Le théorème de Green-Ostrogradski donne: 
    
    \[ P_{eq}=\phi_{elec} \int_{\partial elec} \sigma_{e}(T)\vec \nabla\phi \cdot \vec{n}~dS\]
    \\
    Car le potentiel est constant = $\phi_{elec} $ à la rontière des électrodes. \\
    \subsection{ Expressions des différentes grandeurs et constantes}
    Le taux de conversion $\alpha$ est une sigmoïde : fonction d'erreur de Gauss\\ [0.3cm]
    
    - On modélise la dépendance de la viscosité à la température par une loi de type VFT ( Vogel-Fulcher-Tamman) \\ [0.5 cm]
    
    \centering
    $$
    \eta (T)  = 10^{A_{\eta}} e^{\frac{ln(10) B}{T-T_{\eta}}} ~~~~~~~~~ log(\eta) = A_{\eta} + \frac{B}{T-T_{\eta}}
    $$
    
    
    \raggedright
    - La dépendance de la masse volumique est : \\ [0.5 cm]
    \begin{center}
        
        
        $$
        \rho(T) = \rho_{0}  (1 - \beta \Delta T) ~~~~~~~~~ 
        \beta = -\frac{1}{\rho_{0}} \frac{d\rho}{dt}
        $$
        \\
    \end{center}
    
    
    
    - La dépendance de la conductivité électrique est : \\ [0.5 cm]
    \begin{center}
        $ 
        \sigma_{e} (T) =  A_{e} e^{\frac{-B_{e}}{T}}
        $
    \end{center}
    
    - Dérivée particulaire : ~~~~~~~( importante en méca flu )\\
    
    \begin{center}
        $ \frac{DG}{Dt}=\frac{\partial G}{\partial t} + (\vec {v} \cdot \vec {\nabla } ) G
        $ \\    
    \end{center}
    
    
    
    - Pour calculer le temps  de résidence, à savoir la réponse à un échelon C en entrée :\\ [0.5cm]
    $$
    \phi_{C} =  \oiint_S C\vec{u} \cdot \vec{n}~dS  ~~~~~~~~~~ \phi = \oiint_S \vec{u} \cdot \vec{n}~dS  ~~~~~~~~~~
    \langle C \rangle = \frac{\phi_{C}}{\phi}
    $$ 
    \\ [0.5 cm]
    La distribution des temps de résidence, à savoir la réponse à un dirac, est donnée par \[E(t)=\frac{d\langle C \rangle}{dt}\]
    
    \subsection{État aux frontières}
    
    Les flux thermiques sont donnés par les lois de Newton et Stefan aux parois et surfaces libres. le batch est introduit à iso température. on considère les transfert thermiques de Boltzmann seulement pour la surface libre.
    On peut calculer les flux aux frontières par la loi de Newton sur les transferts conducto-convectifs.\\ 
    \begin{center}
        $$
        \phi_{wall} = h_{wall} (T - T_{\infty})
        $$
    \end{center}
    
    
    
    $T_{haut, min}$ : Température minimale de voûte
    \break
    $\Delta T_{haut}$ : Différence de température maximale


	
	\section{Réalisation des simulations}
 	\raggedright
	Apprentissage du logiciel gmsh pour définir la géométrie d'un four et calculer la maillage qui servira à appliquer la méthode aux éléments finis . Ensuite traduction en format .t , convenant au langage mtc , des fichier gmsh . 
	Connexion au cluster de calcul, prise en main du logiciel Paraview pour visualiser les résultats .
	
	Pour le script, nous avons utilisé un squelette de code fourni par F.Pigeonneau .

    L'objectif est d'utiliser un outil du cemef afin de simuler entièrement un four verrier, et optimiser la forme et l'apport d'énergie pour minimiser la consommation énergétique, les émissions de $CO_{2}$ et maximiser la qualité du verre produit.

    \subsection{Création du four}
    On commence par créer un maillage représentant le four à partir de ses dimensions, qui ont été déterminées préalablement. On utilise l'outil gmesh, qui génère un ensemble de points et de segments les reliants. Tous les champs, scalaires et vectoriels, seront calculés en ces points lors de la simulation. Il est donc nécessaire qu'il y en ait assez pour garantir la précision de la simulation, mais la quantité de calcul augmente avec le nombre de point. Il faut donc trouver la bonne finesse, qui peut être variable dans le maillage. On augmente la densité dans la gorge (tunnel d'évacuation du verre) et à proximité des zones de forte variation dans les champs.\\
    TODO: ajouter images du four (mesh, geométrie)\\
    Une fois le maillage crée, on converti le fichier produit en un format particulier utilisé par la bibliothèque de calcul au moyen d'un script pyhton.

    \subsection{Définition du four pour le calcul}
    Une fois que l'on dispose du maillage du four, et que l'on a décidé des paramètres de la simulation (durée, pas de temps, conditions aux limites, ...) on peut rentrer ces paramètres de façon à ce qu'ils puissent être lus par le logiciel de simulation. Pour cela, on utilise un langage de déclaration créé sur mesure pour l'outil. Tous les champs doivent être initialisés, et la façon dont ils sont calculé à chque pas de temps est défini, au moyen de solveurs d'équations aux dérivées partielles prédéfinis.\\
    TODO: Ajouter extrait de code .mtc\\
    Cette étape est longue et sujette à erreurs, notament à cause du nombre de champs à changer, de l'absence totale de documentation sur l'outil, et du manque de clareté des messages d'erreurs produit par l'outil lors d'un problème dans le sdéclarations.

    \subsection{Calculs de la simulation}
    Pour réaliser les calculs, on utilse le cluster du cemef: un serveur disposant de 100 noeuds de calcul, chacun disposant de deux processeurs pour un total de 64 coeurs. Une fois la simulation lancée, les taches sont réparties de façon automatique entre les coeurs pour une vitesse de calcul maximale. On peut agir sur la simulation au moyen d'un fichier lu à chaque itération qui sert d'interface homme-machine (ihm), et ainsi changer différents paramètres: température d'entrée de la matière dans le four, paramètres de l'asservissement PID, ... De plus, la simulation produit un fichier qui recense la valeur de certains paramètres à chaque itération, comme la température de sortie, les puissances électriques et thermiques fournies et les pertes thermiques à travers les parois, et des fichiers qui enregistre complètement les champs toutes les 100 itérations. On peut alors surveiller l'avancement de la simulation, et agir sur les paramètres afin d'obtenit un résultat satisfaisant (les calculs étant très longs, on veut pouvoir mener à bien une simulation qui n'a pas bien commencer)

    
    \section{Simulations réalisées}
    
	
	\subsection{Première série de calculs}
	\paragraph{}
	 Disposant de 4 postes de calculs , nous avons commencé par chercher à simuler un four  d'une forme ressemblant à celle d'un four industriel .
	 
	 \paragraph{Liste des instructions:}
	 cd /work/MINES-PARISTECH/slamaze
	 sf
	 nano ihm.mtc => changer le temps à 0
	 cd GlassFurnace3D/
	 cimlib CFD driver glassfurnace3d.mtc
	 
	 ls Geometrie/
	 mv Geometrie/indicfrontieres.vtu /work/MINES-PARISTECH/slamaze/ => open avec PARAVIEW, vérifier les frontières
	 nano ihm.mtc , modifier le temps
	 sbatch job.sh

     \subsection{Régulation en puissance à l'aide d'un PID}
     \paragraph{Principe général et Besoin}

     Le paramètre principal à maîtriser pour un four pour un industriel est sa température de sortie. La pilotabilité de l'énergie apportée par électrodes en fait un moyen de régulation particulièrement adapté. Dans ses recherches passées, notre encadrant Franck Pigeonneau a utilisé un régulateur PID agissant sur la puissance des électrode, pour contrôler la température de sortie. La difficulté à modéliser le système formé par le bain de verre réduit la détermination des coefficients du PID à des méthodes empiriques. Nous nous proposons de réaliser une modélisation simplifiée afin de déterminer une estimation plus précise de ces coefficients.

     \paragraph{Modélisation simplifiée}
     Nous avons basé nos calculs sur le modèle de four à 5 électrodes. Puisque les boucles de convections définissent deux "domaines", il est possible de ne considérer que la seconde moitié du four menant à la sortie. Cela mène à ne considérer qu'une seule boucle de convection, qui se modélise par un retard dans le transort de la température entre les différents points du cycle.

     \paragraph{Simulation sur Simlab}
     La boucle de convection est divisée en 2 sections, reliant les 2 zones principales que sont la zone de chauffage par électrodes et la zone de sortie, où le flux de matière est divisé en deux, l'un revenant dans la boucle, et l'un sortant du système. La vitesse de déplacement de la matière dans la boucle est déterminée grâce aux simulations. La zone de chauffage produit l'échauffement de température suivant :
     
     $$\Delta T = \frac{P_{elec}}{\dot{m}\cdot C_p}$$
     
     Reste à modéliser la propagation de la chaleur aux différents points du cycle de convection. Nous pouvons comparer les temps caractéristiques de propagation de la chaleur par convection et par conduction :

     $$\tau_{Conduction} = \frac{a^2\rho C_p}{K} \approx 134 \; h$$
     $$\tau_{Convection} = \frac{d_{parcours}}{u_{moy}} \approx 133 \; s$$

     Cela invite à négliger le terme de conduction thermique, et à modéliser le transport de la température par un retard de valeur $\Delta t=d_{parcours}/u_{moy}$. Néanmoins, la conduction thermique n'étant pas totalement absente, la température n'est pas discontinue, et il convient donc de rajouter un terme pour l'étape du transport, que nous considérerons d'ordre 1 pour plus de simplicité.\\

     \paragraph{Résultats de la simulation}
     La simulation permet d'obtenir une bonne approximation de la valeur des coefficients du régulateur PID. Il apparaît à première vue que le terme le plus important est le terme proportionnel, et qu'il est possible de se passer des corrections intégrale et dérivée. On observe que le système ainsi réglé semble se comporter comme un système d'ordre 1. Theoriquement, il suffirait d'augmenter le terme proportionnel jusqu'à obtenir un dépassement pour diminuer le temps de réponse. Le système réel d'électrode étant limité en puissance, le gain $K_P$ doit lui aussi être limité.     
     Cette première simulation permet de définir les coefficients suivants :
     $$K_P=1000$$
     $$K_I=0.001$$
     $$K_D=0$$
     Le choix de ne pas prendre un terme $K_I$ nul vient du fait que nous n'avons pas pris en compte les pertes thermiques, qui engendrent un besoin en apport de puissance supérieur, compromettant la précision de la régulation limitée au terme proportionnel.

     \paragraph{Simulation du four}
     Afin d'améliorer la précision des coefficients, nous avons simulé un four avec une régulation PID utilisant les coefficients précédemment définis. On observe que la température de sortie $T_{outlet}$ sur laquelle est réalisée l'asservissement converge avec un temps de réponse à 95\% de $2.0\cdot10^4 \; s$. Ce qui est convenable. Néanmoins, comme attendu, la températude de consigne n'est pas atteinte :
     $$T_{finale}=1569.9 \;^{\circ}K ~~~~~~~~~~ T_{Cons}=1593.15 \;^{\circ}K$$
     L'anayse de l'évolution des termes du régulateur permet d'obtenir une nouvelle estimation du coefficient $K_I$

     \paragraph{Nouvelle estimation de $K_I$}
     Une vision du régulateur PID est d'utiliser le terme Intégral pour compenser toutes les pertes énergétiques en régime stationnaire, et le terme proportionnel pour réagir aux fortes variations. En effet, lorsque la température de consigne est atteinte, le terme proportionnel est alors nul. En régime stationnaire, la puissance à apporter au milieu est donnée par : 
    
     $$P_{apport}=\dot{m}C_P(T_{Cons}-T_{Inlet}) + P_{pertes} - P_{gaz}$$
     
     d'où
     $$P_{apport} \approx 25.394\;kW$$

     Grâce à la simulation Cimlib, on calcule la valeur du terme intégral $S_{Int}$ à $t=t_{95\%}$. On obtient alors :
     $$K_I = \frac{P_{apport}}{S_{Int}} \approx 0.039$$
     
	
\end{document}