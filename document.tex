\documentclass[12pt, a4paper, french, BCOR = 0pt, DIV = 10]{scrartcl}
\usepackage{graphicx} % Required for inserting images
\usepackage{babel}
\usepackage[utf8]{inputenc}
\usepackage[T1]{fontenc}
\usepackage{graphicx}
\graphicspath{ {./images/} }

\usepackage{bigints}

\usepackage{pxfonts}

\title{MIG Verre - Modélisation des fours}
\author{\small{Simon Lamaze - Corto Beck - Girardet Grégoire -Fraenkel Paul}}

\begin{document}
	
	\maketitle
	
	\section{Introduction}
	\raggedright
	Ce mini-projet traite de la modélisation et de l'optimisation énergétique des fours à verre.  La compréhension des phénomènes physiques qui prennt place dans un four est essentielle à l'optimisation des procédés. \\ [0.5 cm]
	Voici un four à fusion:
	

	
	
	
	
	\section{ Déroulement de la simulation}
	\raggedright
	Tâche 1 : positionnement du problème : \\
	Conditions limites, conditions initiales, mise en équations \\[0.3 cm]
	Tâche 2: prise en main des outils de calcul\\
	Maillage, cluster, post-traitement, CIMLIB, vitesse des calculs \\ [0.3 cm]
	
	Tâche 3 : Calcul sur un cas à voûte chaude\\ 
	Obtenir un cas convergé, bilan énergétique \\ [0.3 cm]
	
	Tâche 4 : Calcul sur un cas à voûte plus froide \\ 
	Same\\ [0.3 cm]
	
	Tâche 5: Bilan du gain énergétique et des émissions carbone \\ [0.5cm]
	
	
	On  utilise le logiciel Paraview pour visualiser les résultats.
	Visualiser la propagation d'un champ pour définir le temps de résidence.
	Faire plusieurs calculs avec des tailles différentes.
	
	\section{Positionnement du problème}
	- Les équations que nous nous appliquerons à résoudre seront les suivantes : \\
	\subsection{ Equations de Navier-Stokes et thermique}
	- Les équations que nous nous appliquerons à résoudre seront les suivantes : \\
	\centering
	$$
	\left\{
	\begin{array} {ll} 
		\vec {\nabla}. \vec{V} = 0 ~~~~~~~~~  conservation~de~la~masse\\
		
		\rho_{0} \frac{D\vec{V}}{Dt} = -\rho_{0} \beta (T-T_{0}) \vec{g} + \vec {\nabla} . [ 2 \eta (T) \vec{D}] - \vec {\nabla} P  ~~~~~~~~~ conservation~du~moment \\   
		
		\vec{D} = \frac{1}{2} [\vec{\nabla} \vec{V} + (\vec{\nabla} \vec{V})^t ] \\
		\rho [ C_{p}+ \Delta H_{r} \frac{d\alpha}{dT}] \frac{DT}{Dt} = \vec {\nabla} .  (\lambda_{eq}(T) \vec{\nabla} T ) + \sigma_{e}(T) (\vec \nabla\phi)^2  ~~~~~~~~~ \acute equation~de ~la~chaleur \\
		
		
	\end{array}
	\right. 
	$$
	Comme les transfert de chaleur sous formes radiatives sont très importants, on ne peut pas les négliger dans l'équation habituelle de diffusion de la chaleur: $ \rho C_{p} \frac{\partial T}{\partial t} = - \vec{\nabla} . (\lambda\vec{\nabla}T) $. Cependant, comme le milieu est semi-transparent, on peut modéliser l'ensemble des transfert en remplaçant $\lambda$ par $\lambda_{eq} = \lambda + \frac{16n² \sigma T^{3}}{3\beta_{R}} $.
	\\ [0.5 cm]
	
	
	\subsection{Puissance électrique}
	\raggedright
	On se place dans le cas uniphasé, en réalité, dans l'industrie le régime est triphasé, plus de calculs en annexes. \\ [0.2 cm ]
	\centering
	\[ P_{eq}=\int_\Omega \sigma_{e}(T)(\vec \nabla\phi)^2~dV  = \int_\Omega \vec \nabla[\sigma_{e}(T)\vec \nabla\phi]~dV \]\\
	\raggedright
	par intégration par partie et loi de conservation du potentiel ($ \vec{\nabla}^2 \phi = 0 $). Le théorème de Green-Ostrogradski donne: 
	
	\[ P_{eq}=\phi_{elec} \int_{\partial elec} \sigma_{e}(T)\vec \nabla\phi \cdot \vec{n}~dS\]
	\\
	Car le potentiel est constant = $\phi_{elec} $ à la rontière des électrodes. \\
	\subsection{ Expressions des différentes grandeurs et constantes}
	Le taux de conversion $\alpha$ est une sigmoïde : fonction d'erreur de Gauss\\ [0.3cm]
	
	- On modélise la dépendance de la viscosité à la température par une loi de type VFT ( Vogel-Fulcher-Tamman) \\ [0.5 cm]
	
	\centering
	$$
	\eta (T)  = 10^{A_{\eta}} e^{\frac{ln(10) B}{T-T_{\eta}}} ~~~~~~~~~ log(\eta) = A_{\eta} + \frac{B}{T-T_{\eta}}
	$$
	
	
	\raggedright
	- La dépendance de la masse volumique est :
	$$
		\rho(T) = \rho_{0}  (1 - \beta \Delta T) ~~~~~~~~~ 
		\beta = -\frac{1}{\rho_{0}} \frac{d\rho}{dt}
	$$
    $\Delta T = T-T_0$\break
    $T_0, \rho_0$ : (resp.) température et masse volumique de références\\[0.5 cm]
	
	
	
	- La dépendance de la conductivité électrique est :
    $$ 
	   \sigma_{e} (T) =  A_{e} e^{\frac{-B_{e}}{T}}
	$$
    $A_e,B_e$ : Coefficients de la loi d'Arrhénius\break\\[0.5 cm]
	
	- Dérivée particulaire : ~~~~~~~( importante en méca flu )\\
	$$ \frac{DG}{Dt}=\frac{\partial G}{\partial t} + (\vec {v} \cdot \vec {\nabla } ) G
	$$\\[0.5cm]
	
	
	
	- Le calcul tu temps de résidence dans le four est particulièrement important car il conditionne la qualité de la fusion, en réduisant la quantité de bulles ou la présence d'infondus. Une première méthode consiste à étudier la réponse à un Dirac d'apport en matière dans le four en observant le flux de matière à la sortie au cours du temps. Le flux normalisé d'un scalaire C à travers la surface de sortie est donné par :\\[0.3 cm]
	$$
	\phi_{C} =  \oiint_S C\vec{u} \cdot \vec{n}~dS  ~~~~~~~~~~ \phi = \oiint_S \vec{u} \cdot \vec{n}~dS  ~~~~~~~~~~
	\langle C \rangle = \frac{\phi_{C}}{\phi}
	$$ 
	\\ [0.5 cm]
	Néanmoins, il est plus aisé de calculer la réponse à un échelon, ce qui implique que la distribution des temps de résidence, à savoir la réponse à un dirac, est alors donnée par \[E(t)=\frac{d\langle C \rangle}{dt}\]
	
	\subsection{État aux frontières}
	
	Les flux thermiques sont donnés par les lois de Newton et Stefan aux parois et surfaces libres. Le batch (matière d'entrée) est introduit à iso température. On considère les transfert thermiques de Boltzmann uniquement pour la surface libre.
	On peut calculer les flux aux frontières par la loi de Newton sur les transferts conducto-convectifs.\\ 
	\centering
	$$
	\phi_{wall} = h_{wall} (T - T_{\infty}) ~~~~~~~~	
	\phi_{haut} = h_{haut} (T - T_{haut}) + \epsilon \sigma (T^4 - T_{haut}^4)
	$$
    \raggedright
    $\epsilon : $\break
    $\sigma_{SB}$ : Constante\ de\ Stefan-Boltzman
    \\[0.5 cm]
    Il est possible de définir la température $T_{haut}$ comme une constante sur l'ensemble de la surface libre. Néanmoins, le chauffage par flamme dans un four industriel présente une hétérogénéité de cet apport calorifique, avec généralement une température plus important au centre. Bien qu'il serait théoriquement possible de réaliser une simulation CFD de la dynamique du gaz lors de sa combustion, ceci est inenvisageable à cause du temps de calcul nécessaire. Il convient donc d'approximer cette répartition de température par une fonction constante du temps.
    $$
        T_{haut}(\vec{r}) = T_{haut, min} + \Delta T_{haut} \space e^{-||\vec{r^{\prime}}||}
    $$
    $T_{haut, min}$ : Température minimale de voûte
    \break
    $\Delta T_{haut}$ : Différence de température maximale
	
	\section{Prise en main du langage et des logiciels de visualisation}
 	\raggedright
	Apprentissage du logiciel gmsh pour définir la géométrie d'un four et calculer la maillage qui servira à appliquer la méthode aux éléments finis . Ensuite traduction en format .t , convenant au langage mtc , des fichier gmsh . 
	Connexion au cluster de calcul, prise en main du logiciel Paraview pour visualiser les résultats .
	
	Pour le script, nous avons utilisé un squelette de code fourni par F.Pigeonneau .
	
	\section{Simulations réalisées}
	
	\subsection{Première série de calculs}
	\paragraph{}
	 Disposant de 4 postes de calculs , nous avons commencé par chercher à simuler un four  d'une forme ressemblant à celle d'un four industriel .
	 
	 \paragraph{Liste des instructions:}
	 cd /work/MINES-PARISTECH/slamaze
	 sf
	 nano ihm.mtc => changer le temps à 0
	 cd GlassFurnace3D/
	 cimlib CFD driver glassfurnace3d.mtc
	 
	 ls Geometrie/
	 mv Geometrie/indicfrontieres.vtu /work/MINES-PARISTECH/slamaze/ => open avec PARAVIEW, vérifier les frontières
	 nano ihm.mtc , modifier le temps
	 sbatch job.sh
	 
 
	
	
\end{document}
