\documentclass[12pt, a4paper, french, BCOR = 0pt, DIV = 10]{scrartcl}
\usepackage{graphicx} % Required for inserting images
\usepackage{babel}
\usepackage[utf8]{inputenc}
\usepackage[T1]{fontenc}
\usepackage{graphicx}
\graphicspath{ {./images/} }

\usepackage{bigints}

\usepackage{pxfonts}

\title{MIG Verre - Modélisation des fours}
\author{\small{Simon Lamaze - Corto Beck - Girardet Grégoire -Fraenkel Paul}}

\begin{document}
	
	\maketitle
	
	\section{Introduction}
	\raggedright
	Ce mini-projet traite de la modélisation et de l'optimisation énergétique des fours à verre.  La compréhension des phénomènes physiques qui prennt place dans un four est essentielle à l'optimisation des procédés. \\ [0.5 cm]
	Voici un four à fusion:
	

	
	
	
	
	\section{ Déroulement de la simulation}
	\raggedright
	Tâche 1 : positionnement du problème : \\
	Conditions limites, conditions initiales, mise en équations \\[0.3 cm]
	Tâche 2: prise en main des outils de calcul\\
	Maillage, cluster, post-traitement, CIMLIB, vitesse des calculs \\ [0.3 cm]
	
	Tâche 3 : Calcul sur un cas à voûte chaude\\ 
	Obtenir un cas convergé, bilan énergétique \\ [0.3 cm]
	
	Tâche 4 : Calcul sur un cas à voûte plus froide \\ 
	Same\\ [0.3 cm]
	
	Tâche 5: Bilan du gain énergétique et des émissions carbone \\ [0.5cm]
	
	
	On  utilise le logiciel Paraview pour visualiser les résultats.
	Visualiser la propagation d'un champ pour définir le temps de résidence.
	Faire plusieurs calculs avec des tailles différentes.
	
	\section{Positionnement du problème}
	- Les équations que nous nous appliquerons à résoudre seront les suivantes : \\
	\subsection{ Equations de Navier-Stokes et thermique}
	- Les équations que nous nous appliquerons à résoudre seront les suivantes : \\
	\centering
	$$
	\left\{
	\begin{array} {ll} 
		\vec {\nabla}. \vec{V} = 0 ~~~~~~~~~  conservation~de~la~masse\\
		
		\rho_{0} \frac{D\vec{V}}{Dt} = -\rho_{0} \beta (T-T_{0}) \vec{g} + \vec {\nabla} . [ 2 \eta (T) \vec{D}] - \vec {\nabla} P  ~~~~~~~~~ conservation~du~moment \\   
		
		\vec{D} = \frac{1}{2} [\vec{\nabla} \vec{V} + (\vec{\nabla} \vec{V})^t ] \\
		\rho [ C_{p}+ \Delta H_{r} \frac{d\alpha}{dT}] \frac{DT}{Dt} = \vec {\nabla} .  (\lambda_{eq}(T) \vec{\nabla} T ) + \sigma_{e}(T) (\vec \nabla\phi)^2  ~~~~~~~~~ \acute equation~de ~la~chaleur \\
		
		
	\end{array}
	\right. 
	$$
	Comme les transfert de chaleur sous formes radiatives sont très importants, on ne peut pas les négliger dans l'équation habituelle de diffusion de la chaleur: $ \rho C_{p} \frac{\partial T}{\partial t} = - \vec{\nabla} . (\lambda\vec{\nabla}T) $. Cependant, comme le milieu est semi-transparent, on peut modéliser l'ensemble des transfert en remplaçant $\lambda$ par $\lambda_{eq} = \lambda + \frac{16n² \sigma T^{3}}{3\beta_{R}} $.
	\\ [0.5 cm]
	
	
	\subsection{Puissance électrique}
	\raggedright
	On se place dans le cas uniphasé, en réalité, dans l'industrie le régime est triphasé, plus de calculs en annexes. \\ [0.2 cm ]
	\centering
	\[ P_{eq}=\int_\Omega \sigma_{e}(T)(\vec \nabla\phi)^2~dV  = \int_\Omega \vec \nabla[\sigma_{e}(T)\vec \nabla\phi]~dV \]\\
	\raggedright
	par intégration par partie et loi de conservation du potentiel ($ \vec{\nabla}^2 \phi = 0 $). Le théorème de Green-Ostrogradski donne: 
	
	\[ P_{eq}=\phi_{elec} \int_{\partial elec} \sigma_{e}(T)\vec \nabla\phi \cdot \vec{n}~dS\]
	\\
	Car le potentiel est constant = $\phi_{elec} $ à la rontière des électrodes. \\
	\subsection{ Expressions des différentes grandeurs et constantes}
	Le taux de conversion $\alpha$ est une sigmoïde : fonction d'erreur de Gauss\\ [0.3cm]
	
	- On modélise la dépendance de la viscosité à la température par une loi de type VFT ( Vogel-Fulcher-Tamman) \\ [0.5 cm]
	
	\centering
	$$
	\eta (T)  = 10^{A_{\eta}} e^{\frac{ln(10) B}{T-T_{\eta}}} ~~~~~~~~~ log(\eta) = A_{\eta} + \frac{B}{T-T_{\eta}}
	$$
	
	
	\raggedright
	- La dépendance de la masse volumique est : \\ [0.5 cm]
	\begin{center}
		
		
		$$
		\rho(T) = \rho_{0}  (1 - \beta \Delta T) ~~~~~~~~~ 
		\beta = -\frac{1}{\rho_{0}} \frac{d\rho}{dt}
		$$
		\\
	\end{center}
	
	
	
	- La dépendance de la conductivité électrique est : \\ [0.5 cm]
	\begin{center}
		$ 
		\sigma_{e} (T) =  A_{e} e^{\frac{-B_{e}}{T}}
		$
	\end{center}
	
	- Dérivée particulaire : ~~~~~~~( importante en méca flu )\\
	
	\begin{center}
		$ \frac{DG}{Dt}=\frac{\partial G}{\partial t} + (\vec {v} \cdot \vec {\nabla } ) G
		$ \\    
	\end{center}
	
	
	
	- Pour calculer le temps  de résidence, à savoir la réponse à un échelon C en entrée :\\ [0.5cm]
	$$
	\phi_{C} =  \oiint_S C\vec{u} \cdot \vec{n}~dS  ~~~~~~~~~~ \phi = \oiint_S \vec{u} \cdot \vec{n}~dS  ~~~~~~~~~~
	\langle C \rangle = \frac{\phi_{C}}{\phi}
	$$ 
	\\ [0.5 cm]
	La distribution des temps de résidence, à savoir la réponse à un dirac, est donnée par \[E(t)=\frac{d\langle C \rangle}{dt}\]
	
	\subsection{État aux frontières}
	
	Les flux thermiques sont donnés par les lois de Newton et Stefan aux parois et surfaces libres. le batch est introduit à iso température. on considère les transfert thermiques de Boltzmann seulement pour la surface libre.
	On peut calculer les flux aux frontières par la loi de Newton sur les transferts conducto-convectifs.\\ 
	\centering
	$$
	\phi_{wall} = h_{wall} (T - T_{\infty}) ~~~~~~~~	
	\phi_{haut} = h_{haut} (T - T_{haut}) + \epsilon \sigma (T^4 - T_{haut}^4)
	$$
	
	\section{Prise en main du langage et des logiciels de visualisation}
 
	
	
	
\end{document}
